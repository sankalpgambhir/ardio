\subsection{Doobit™}
\label{sec:doobit}

Working on the Arduino with signals very quickly turned into a constant battle
of resolution and storage, limited by its relatively tiny memory. After msot
optimizations we went through, managing memory manually very closely and
ensuring sequentialized execution contexts, we were teribly bottlenecked by the
storage. To work around this physical limit, we manage each byte of the stored
signal manually and instead of one, store two signal data points in every byte.
This reduces our data resolution by way of being limited to 4 bits, but grants
us unmatched robustness to noise by comparison, by doubling possible data
processing capabilities. The system has been affectionately named Doobit.

\begin{lstlisting}[language=C++]
// bit mask storage
struct doobit{
    uint_fast8_t data;

    doobit(int8_t x = 0, int8_t y = 0){ // handles all our casts too
        this->storelow(x);
        this->storehigh(y);
    }

    void storelow(int8_t);
    void storehigh(int8_t);

    int8_t getlow();
    int8_t gethigh();

    int16_t operator*(doobit& b);
};
\end{lstlisting}

A \texttt{struct} provides us fast access with very little memory and
performance overhead, something that only becomes more and more negligible as we
increase our (now doubled!) data numbers.

Unpacking the code block, \texttt{data} is the actual storage, an unsigned 8-bit
type, chosen this way to avoid accidental signed interpretation and any unwanted
processing by the compiler. Reliance on unsigneds in a case such as this is
common even within the compiler itself, where it would convert signeds to
unsigneds before evalutation to avoid ambiguity. In particular, a two's
complement could scramble the data beyond recognition quite quickly.

The \texttt{fast} part of \texttt{uint\_fast8\_t} indicates to the compiler that
we are looking for a type that is atleast 8 bits, but is the fastest among
those. On an Arduino Uno, of course, this is just the 8 bit unsigned integer,
but on more exotic embedded systems this could end up being a 9 or 10 bit type,
if not more. This notation allows for some compatibility between systems, though
as is with embedded systems, one would try to create more efficient structures
that exploit the architecture of those systems.

The storage of the signal is handled by the \texttt{storelow()} and
\texttt{storehigh()} functions, which store data into the 4 least significant
and 4 most significantbits of the storage \texttt{data} respectively. We look at
one of the functions:

\begin{lstlisting}[language=C++]
void doobit::storelow(int8_t x){
	this->data &= 0b11110000;  // clear for storage

	x += 7; // remove signed component
	assert((!(x & 0b11110000)) && "doobit range violation");

	this->data |= x;
}
\end{lstlisting}

This function takes in a value \texttt{x}, to be stored in the lower 4 bits of
\texttt{data}, and preparing for it, clears the lower 4 bits via an \texttt{AND}
operation with a bitmask \texttt{11110000}. Following this, a manual conversion
is made to ensure \texttt{x} is unsigned. The operation moves \texttt{x}'s
previous range, -7 to +7, to now 0 to 14, with 15 remaining generally unused,
rushing to somehow occupy this position leads to little benefit and after much
trial to squeeze extra storage out of the system, was abandoned. 

The \texttt{assert} exists merely for testing purposes to ensure our data can
indeed fit in 4 bits. Finally, having ensured \texttt{x} has only its lower bits
populated, an \texttt{OR} operation with the storage inserts it in.

The \texttt{storehigh()} function works in a similar manner, albeit with a bit
shift and an inverted bit mask to work on the 4 higher bits.

Now remains the issue of retrieving data from this storage, and is done as very
much the inverse of how it is stored:

\begin{lstlisting}[language=C++]
int8_t doobit::getlow(){
    int8_t x = (data & 0b00001111); // bitmask
    return (x-7); // reinsert sign
}
\end{lstlisting}

The \emph{unrequired} data is removed via a bit mask, and would be moved
rightwards via a bit shift in the case of \texttt{gethigh()}, and its signed
nature is restored by shifting it back to the original range.

The conversion to unsigned is quite important to have to not manage the carry
bits arising from a two's complement operation. The number -7 in C++ could
ambiguously be coming from an \texttt{int} (internally \texttt{int32\_t}) in
which case the signed bit is the 32nd, while it could also be coming from an 8
or 16 bit type, making the signed bit unclear and its extraction slow and
painful. Asking for a change of range was the fastest of the operations we
tested, included several bitwise only operations.

The storage could be optimized for several arithmetic operations, but for our
case in particular, for the correelation setups, we need only multiplication. As
of now, this is done simply by retrieving the numbers individually and
multiplying them pairwise. This is done as opposed to attempting bitwise
operations as (1), multiplication operations are quite optimized on a circuit
level in modern processors anyway, and more importantly (2), 4 bit being such a
limited storage type, would cause an overflow for most possible multiplication
operands.

\begin{lstlisting}[language=C++]
int16_t operator*(doobit& b){
    auto highprod = this->gethigh() * b.gethigh();
    auto lowprod = this->getlow() * b.getlow();
    return (highprod + lowprod);
}
\end{lstlisting}

This definition is obviously not compatible with all arithmetic operations, but
quite specific for our limited operations, kept this way to prevent
overcomplication of the rather heavily utilized routine.

The full code for all the functions may as always be found
\hyperref[sec:arduinocode]{in the Appendix}.