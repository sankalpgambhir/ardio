
The main idea of this project is to extract some characteristic audio data from
the signal without having to resort to memory and computationally expensive
fourier transforms. The 2KiB SRAM of the Arduino is the biggest botleneck here
for doing any processing. In order to do this processing, we came up with
multiple optimizations. 

\subsection{Optimizations}

\subsubsection{Cross Correlations} 
Instead of doing a FFT on the Data, we will extract a few characteristic
frequencies components by correlating the signal with a set of frequencies. The
expression for crosscorrelation of 2 discrete signals \(x,y\) is given by

\begin{equation}
    R_{xy}[k] = \sum_{i} x[i]y[i-k]
\end{equation}

Along with some normalization. The harmonics form a linearly independent set and
return 0 correlation when the product is integrated over several time periods,
i.e.

\begin{equation}
    \frac{1}{\pi} \int_0^{2\pi}sin(n_1x)sin(n_2x) = \delta_{nn'}  
\end{equation} 

Where \(\delta\) is the usual Kronecker delta. We can normalize the correlation
with the autocorrelation at 0 to get a number between -1 and 1 that
characterized the coeficients.

\begin{equation}
    c_{xy} = \frac{R_{xy}[0]}{\sqrt{R_{xx}[0]R_{yy}[0]} }
\end{equation} 

However, the above expression does not account for the phase shift \(\phi\)
between the harmonics which reduces the correlation by a factor of
\(cos(\phi)\)Therefore we modify check the signal with phase shifted test
harmonics by modifying the expression to:

\begin{equation}
    c_{xy} = \frac{\mathnormal{max}\{R_{xy}[k]\}}{\sqrt{R_{xx}[0]R_{yy}[0]} }
\end{equation}

This ensures that the loss due to potential phase shift is avoided by getting an
estimate of the phase. Further, these correlations avoid the complex
multiplications required in FFT calculations while giving more characteristic
data. 

\subsubsection{Memory Management}
We store the signal using an array of 4 bit numbers, ie numbers from -7 to 7.
This has several benefits: a)You can store more numbers b)You can perform the
calculations in a smaller time period as the Arduino processor is 8-Bit. Instead
of requiring multiple clock cycles to process 16 / 32 Bit datatypes like float
we can process the signals much faster 

\subsubsection{Frequency Space Pruning}
Beginning with a dirac comb in frequency space, we prune a lattice tree of the
frequency power set via depth first binary search. This bypasses the
computationally expensive cross correlation calculations for a lot of (wrong)
frequencies. 