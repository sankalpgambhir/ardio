\subsection{Motivation}

\subsubsection{Why try to make better Fourier Transforms?} Fourier transforms
are the most important tools used in data processing. The relative amplitudes of
the signal at different frequencies give us a lot of characterising data of the
signal, and changing the signal in frequency space has a lot of applications,
like auto-tuning, removing noise etc. 

\begin{equation}
    \tilde{f}(\omega) = \int_{-\infty}^{\infty} e^{-i\omega t}f(t) dt
\end{equation}

\subsubsection{Fourier is Overkill.}
Looking at the structure of the Fourier transform, it is easy to see that it is
closely resembles the idea of a correlation, essentially extracting from the
signals its \emph{similarity} to a given frequency. These coefficients, gathered
using all frequencies, can produce a one to one mapping to a function space in
the frequency realm. However, the entire function is far too much information.
For most tasks involving categorization and identification of sounds,
fingerprinting is more than good enough. That is, considering the delta response
of the system, its restriction to a tiny subset of points in frequency space.

As such, simply extracting the info for these frequencies alone is sufficient,
and can be done whilst incorporating several statistical approximations. This is
easily done with a correlation of the signal with a pure mode. 

However, when frequencies are aggregated, the phase difference is absorbed into
the coefficients of multiple frequencies, but in the correlation scenario, no
such hope exists. This is where cross-correlation comes in. 

\begin{equation}
    (f \ast g)(t) \triangleq \int_{-\infty}^{+\infty} \overline{f(\tau)} \cdot g(t+\tau) d\tau
\end{equation}

By considering the similarities of shifted signals, we can infer more precise
information about their similarity in space, by completely disregarding the
temporal dimension. This is immensely useful in matching an externally sourced
wave of unknown phase with a synthetic mode of known or unknown phase to
identify the wave.

In particular, the global maximum of the cross-correlation may be taken as the
phase independent correlation of two signals.

\begin{equation}
    corr(f, g) \triangleq \max((f \ast g) (t))_t
\end{equation}
