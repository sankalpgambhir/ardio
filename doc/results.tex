\subsection{Synthetic Data}

First, tests were run on data generated in situ from a list of frequencies, coefficients, and phase differences

\begin{align*}
    c = \{c_1, c_2, \ldots, c_n\}\\
    \omega = \{\omega_1, \omega_2, \ldots, \omega_n\}\\
    \phi = \{\phi_1, \phi_2, \ldots, \phi_n\}\\
\end{align*}

which generate a 'generating' function for the discrete signal later

\begin{align*}
    f(x) = \sum_{i = 1}^{n} c_i \cdot sin(w_i x + \phi_i)
\end{align*}

implemented as a lambda-function in code 

\begin{lstlisting}[language=C++]
auto f_gen = [w, c, p](int x){
                float sum = 0;
                for(int i = 0; i < wlist.size(); i++){
                    sum += c[i] * sin(w[i]*x + p[i]);
                }
                return 7.0*sum/float(std::accumulate(c.begin(), c.end(), 0));
            };
            
auto f = new signal[SIZE];

for(int x = 0; x < SIZE; x++){
    f[i] = signal(f_gen(2*x), f_gen(2*x + 1));
}
\end{lstlisting}

The normalising factor may be succinctly written as 

\begin{align*}
    a_N = \frac{7.0}{\sum_{i = 1}^{n}c_i}~.
\end{align*}

It simply ensures that the cross correlation of the signals do not blow up and
that a single signal agnostic threshold value may be chosen as a metric for
matching. The factor of 7 scales the float value to the range (-7, 7) so as to
maintain reasonable resolution after conversion to integer type to save memory.

The signal is generated as even/odd pairs to facilitate pairwise storage
implemented by \nameref{sec:doobit}.

\begin{figure}[ht]
    \centering
    \def\arraystretch{1.5}
    \setlength\tabcolsep{2em}
    \begin{tabular}{c | c | c}
        NCC Threshold (t)   & Input frequencies and coefficients (kHz)      & Output \\ \hline
        0.1                 & 0.3, 0.5, 0.8                                 & 0.3, 0.5, 0.8 \\
        0.1                 & 0.3, 0.5, 0.8$^\dagger$                       & 0.3, 0.5, 0.8 \\
        0.1                 & 0.3, 0.5$^\dagger$, 0.8$^\dagger$             & 0.3, 0.5, 0.8 \\
        0.2                 & 0.3, 0.5$^\dagger$, 0.8$^\dagger$             & 0.3, 0.5, 0.8 \\
        0.2                 & 0.3, 0.5$^\dagger$, 0.8$^\dagger$             & 0.3, 0.5, 0.8 \\
        0.2                 & 0.3, 4$\times$ 0.5$^\dagger$, 0.8$^\dagger$   & 0.3, 0.5, 0.8 \\
        0.4                 & 0.3, 4$\times$ 0.5$^\dagger$, 0.8$^\dagger$   & 0.3, 0.5, 0.8 \\
        0.5                 & 0.3, 4$\times$ 0.5$^\dagger$, 0.8$^\dagger$   & 0.3, 0.5      \\
        0.6                 & 0.3, 4$\times$ 0.5$^\dagger$, 0.8$^\dagger$   & 0.5           \\
        0.8                 & 0.3, 4$\times$ 0.5$^\dagger$, 0.8$^\dagger$   & $\emptyset$   \\
        0.1                 & 0.3, 0.5, 0.35$^\dagger$                      & 0.3, 0.4, 0.5 \\
        0.2                 & 0.3, 0.5, 0.35$^\dagger$                      & 0.3, 0.5     
    \end{tabular}
    \captionsetup{justification=centering}
    \caption{Experiments in varying synthetic inputs and thresholds.\\
    $\dagger.$ Phase shifted. Weight unit unless otherwise specified}
    \label{fig:synthexp}

\end{figure}

screenshots of code compiled and running on arduino

photo video uploads